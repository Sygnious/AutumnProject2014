\chapter{Conclusion}

The performance gains attained from using an accelerator in conjunction with a DMA
proves the usefulness of using an accelerator for SHA-256 hashing. Although the
performance seen when enabling the DMA was not as expected due to the addressing
requirements of the AXI4 bus used in the test system, it still allows the CPU to
do other work or enter a sleep state while memory is being transferred, thus
the DMA is important to the energy efficiency of the system.

With the modules inserted into SHMAC, a full bitcoin mining system can be
constructed that can take advantage of power savings from not having to do
the hashing and data transfers involved in software. Using SHMAC for mining
bitcoins will possibly be the very first time a heterogeneous architecture
is used for bitcoin mining.

\section{Future Work}
\label{sec:future-work}

\subsection{Simplified DMA Designs} 
In order for the design to be more appropriate for environments where there are
constraints on, for example, logic resources or size of the module, certain simplifications
are possible.

\todo{Consider making diagrams to show the alternatives}

\subsection{Reducing the Controller Logic}
It is possible to remove some of the main components from the DMA module and still
have a functional DMA module. The request FIFO and the DMA controller could be removed,
and a channel could be directly connected to the bus interface adapter with little
extra work.

This would slightly reduce the number of cycles that is needed to process a request.
The system would also be smaller, including less combinatorics and registers.

On the other hand, this makes the DMA module much more dependent on the external system, since requests are processed and set externally, not by the DMA module itself.
Futhermore, removing the controller also constricts the possibility for the DMA module itself to be able to process different types of requests.
In the minimum case, the only requests that it needs to handle are data transfer requests.
But if the DMA module is implemented in the SHMAC system, and one wants an as efficient data transfer as possible, one have to consider the option of dynamicly forwarding the request.
An example would be forwarding to the DMA module that is \todo{Mentioned in chapter 3.5.1., see figure 3.3} closest to both source and destination.

Part of the purpose with the design is to make the DMA module generic, so that it can implemented in different systems.
In order to make the design as generic as possible, a controller and an incoming request FIFO
in case of multiple requests is therefore essential.

\subsection{Using Channels with a Private Data Buffer}
Instead of using a channel with shared data buffer, where the data output owned by
one of the channels is sent straight through the arbiter, it could be sent to a
private buffer inside the correct channel.
\todo{An explanation of the shared/private buffers will need to be added somewhere}

\todo{Expand when you wake up from bed. Good night}
Pros: Secure data

Cons: More unnecessary combinatorics + no better throughput


