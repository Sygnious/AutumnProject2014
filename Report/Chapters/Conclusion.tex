\chapter{Conclusion}

The performance gains attained from using an accelerator in conjunction with a DMA
proves the usefulness of using an accelerator for SHA-256 hashing. Although the
performance seen when enabling the DMA was not as expected due to the addressing
requirements of the AXI4 bus used in the test system, it still allows the CPU to
do other work or enter a sleep state while memory is being transferred, thus
the DMA is important to the energy efficiency of the system.

With the modules inserted into SHMAC, a full bitcoin mining system can be
constructed that can take advantage of power savings from not having to do
the hashing and data transfers involved in software. Using SHMAC for mining
bitcoins will possibly be the very first time a heterogeneous architecture
is used for bitcoin mining, and will nevertheless provide useful insight
into how complex software can benefit from running on a heterogeneous
architecture.

\section{Future Work}
\label{sec:future-work}

Although the project shows the improvements obtained using an accelerator
for hashing and data transfer, further improvements can be obtained.

\subsection{Optimizing the Hashing Accelerator}
By optimizing the hashing accelerator, it is possible to create an accelerator
that can run on higher clock speeds, increasing the throughput of a system that
is dependent on high-speed hashing, such as a bitcoin mining system.

\subsection{Simplifying the DMA design} 
In order for the design of the DMA to be more appropriate for environments where
there are constraints on, for example, logic resources, certain simplifications
are possible.

It is possible to remove some of the main components from the DMA module and still
have a functional DMA module. The request FIFO and the DMA controller could be removed,
and a channel could be directly connected to the bus interface adapter with little
extra work.

This would slightly reduce the number of cycles that is needed to process a request.
The system would also be smaller, including less combinatorics and registers.

On the other hand, this makes the DMA module much more dependent on the external system, since requests are processed and set externally, not by the DMA module itself.
Futhermore, removing the controller also constricts the possibility for the DMA module itself to be able to process different types of requests.
In the minimum case, the only requests that it needs to handle are data transfer requests.
But if the DMA module is implemented in the SHMAC system, and one wants an as efficient data transfer as possible, one have to consider the option of dynamicly forwarding the request.
An example would be forwarding to the DMA module that is \todo{Mentioned in chapter 3.5.1., see figure 3.3} closest to both source and destination.

Part of the purpose with the design is to make the DMA module generic, so that it can implemented in different systems.
In order to make the design as generic as possible, a controller and an incoming request FIFO
in case of multiple requests is therefore essential.

\subsection{Improving the Bus Network}
Since the bus network has been determined to be the bottleneck of the system, different topologies
can provide a less congested bus between memory or the hashing accelerator. Different topologies
would have to be investigated, but one option is to place the hashing module and some memory together
with a DMA on a private bus separate from the rest of the system could allow these modules to
communicate without having to deal with interference from other sources of bus traffic.

However, in a heterogeneous system such as SHMAC, options that might work well in a shared-bus
system might not provide the same benefits, so an analysis would have to be made in the context
of SHMAC as well.

