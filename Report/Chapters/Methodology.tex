\chapter{Methodology}

\section{Hasing module}

\subsection{Test Setup}
In order to evaluate the performance increase obtained when using a dedicated
hashing accelerator, a test system was constructed using Xilinx' Vivado
development studio.

\begin{figure}[ht]
	\includegraphics[width=0.95\textwidth]{Figures/testsystem-vivado.png}
	\caption{Overview of the test system}
	\label{fig:testsystem-vivado}
\end{figure}

The test system is controlled by a MicroBlaze microprocessor, configured to
include the optional barrel shifter, integer divider and pattern comparator.
In order to run the XilKernel real-time operating system, the system also
includes a simple timer module for use by the kernel.

A UartLite module was included to provide I/O in order to communicate with
the system from a desktop computer, as well as a GPIO module for additional
debugging use.

The system runs on a 50~MHz clock which is synthesized by a clock generator
module from a 100~MHz input clock.

In order to run performance tests on the system, the DMA described in
section \ref{undefined}\todo{Insert correct reference here} and four
SHA256 hashing modules are included.\todo{Maybe reduce to one now}
A fixed-interval timer, that generates an interrupt every second, is
included in order to calculate the performance per second.

\subsection{Test Software}
In order to test the hashing modules and how the performance scales when including
multiple hashing modules, a benchmark application was written. The benchmark
runs a specified number of threads that tries to hash a single block of data
over and over again.

A simple scheduler is used to provide each thread with an available hashing module.
If no module is available, the thread blocks on a semaphore until a module is available.

Once a second, the number of completed hashes for each module is added up and
reported over the UART.

The test software is also designed so that it can be run with a software implementation
of the hashing algorithm instead of the hashing modules. This makes it possible
to compare the performance when using an accelerator as compared to not using
an accelerator.

\subsection{Measurements and Benchmarks}
The most important performance measure for a Bitcoin mining system is the number
of hashes per second it can sustain. Therefore, once a second the number of hashes
computed since the previous second is calculated and sent over the UART.

Another important measurement is how the inclusion of a DMA affects the performance
of the system.

\subsection{Measuring DMA Performance}

In the test system, a load takes at least 2 cycles, and a store takes at least 3 cycles.
Branching and incrementation takes at least 1 cycle each.
For M data, the total transfer for the CPU takes at least 
\\ (3 + 2 + 1 + 1) * M = 7M cycles, for the loop only.
\\ If a DMA module is present, it can relieve the CPU for this work, letting the CPU focus on another work, saving it 7M cycles for each transfer, with exception of overhead, such as activating the DMA Module and handling interrupts when done.
In addition, a well designed DMA Module may have less overhead, enabling better \todo{With more than one channel active, a transfer does not necessarily become faster by itself} throughtput.

A program with M data to be transfered will be used, and the number of clock cycles spent on other work than data transfer for the CPU will be measured.
A comparison with running with and wihtout DMA Module will be done, to see the CPU can spend at least 7M more cycles of work on other tasks.
This does not include data loads/stores the CPU needs for itself.

In addition, the hashing module will be tested with using the DMA Module for data transfer, to see if adding DMA Module improves the overall performance for the hashing module.