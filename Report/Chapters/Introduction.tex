\chapter{Introduction}

\section{The Bitcoin Currency}
Bitcoin is a decentralized currency using a peer-to-peer network to replace
the financial institutions that are used to process transactions in conventional
currency systems.

The network keeps a ledger of all transactions in a linked list of blocks, called the
block-chain. Each block contains a list of all transactions executed since the last
block was published. Blocks are generated periodically through a process
commonly known as ``mining''. This is due to the fact that a valid block must satisfy
the requirement that the hash of the block header must begin with a certain number of
zero bits, which is determined by a variable difficulty value used to limit the number
of blocks being generated each hour\cite{bitcoin}.

As an incentive to keep generating new blocks, a reward is offered on each new block
generated. This reward is currently at 25~bitcoins (467~\$ (9.11.2014))\todo{Explain why, also fix date format}.

\todo{There are many more details to write about how bitcoins work, but it's boring stuff.}

\subsection{Mining Bitcoins}
The process of creating a new block in the bitcoin network is called ``mining''. The basic
principle of bitcoin mining is to create a block header that generates a SHA-256 hash with
a value lower than a preset target value.

The search for a valid block hash has to be done by brute-force search. The block header
contains several fields that can be varied to produce different hashes.

Mining performance is measured in hashes per second, denoted H/s.

Some keywords for later:

-new blocks

-algorithm

-network

-getwork + getworktemplate (if neccessary)

\subsection{Pooled Mining}
Due to the increasing difficulty of creating new bitcoin blocks, it has become impractical
to mine bitcoins with either commodity hardware or even specialized ASIC miners. According
to \texttt{blockchain.info}, a website tracking the bitcoin blockchain and network, the
total number of hashes per second in the bitcoin network is about 214 EH/s\footnote{exahashes per second, $10^18$ H/s}.
In comparison, a single modern, high-end GPU has a hashing performance of about 800 MH/s.

In order for anyone with slower hardware to still be able to earn bitcoins through mining,
bitcoin pools were created. A mining pool allows large numbers of people to cooperate on
finding new bitcoin blocks. When a new valid block is found, the reward is split between
the participants based on how much work they contributed towards finding the block.

- stratum protocol

\subsubsection{The Stratum Protocol}


