\chapter{Introduction}

This report details an exploration of the possible benefits of accelerating
<<<<<<< HEAD
SHA-256 hashing using an accelerator module and a DMA\footnote{\textbf{D}irect \textbf{M}emory
\textbf{A}ccess} Engine, for possible future inclusion in the SHMAC\footnote{\textbf{S}ingle-ISA \textbf{H}eterogenous \textbf{Ma}ny-core \textbf{C}omputer} research computer, which is a
heterogeneous architecture consisting of different processing elements with
optional accelerators. SHMAC is further described in section \ref{sec:shmac}.

\section{Original Assignment Text}

This project aims to develop an accelerator for a multi-core heterogeneous
computing platform, with the aim of eventually instantiating a complete hardware-based bitcoin miner.
Bitcoin mining is, at its core, a SHA-256 hashing problem, so part of the
assignment will be to make the hashing interface generic enough such that other
cryptographic algorithms can be readily developed. Furthermore, various constraints
on the heterogeneous architecture may necessitate the implementation of new
features, for example, direct memory access (DMA).

Important parts of this project includes reviewing literature in order to discover
possible benefits of a bitcoin miner in the context of a multi-core heterogeneous
computing platform, implementing a SHA-256 accelerator prototype and testing it on
an FPGA, evaluating the energy efficiency of the accelerator as compared to a
general purpose CPU, implementing a DMA and evaluating the impact of it in the context
of SHMAC.

\section{Bitcoin Mining as a Model Problem}

Bitcoin is a virtual currency that is produced through an algorithm that
requires repeatedly hashing data structures with the SHA-256 algorithm
in order to produce a target hash value. There are more details about
bitcoins and how they are created in section \ref{sec:bitcoins}.

Bitcoin mining is a problem that there exists many different accelerators
for, implemented both in software and hardware \cite{bespoke-silicon}.
There is also an established performance metric for bitcoin mining hardware:
hashes per second, often abbreviated H/s. This metric makes it easy
to measure and compare the performance of our accelerator.

\section{Accelerators and Heterogeneous Systems}

A heterogeneous processor architecture is an architecture that contains multiple
cores of different sizes and capabilities. For an overview of why heterogeneous
architectures may become increasingly important in the future, see section
\ref{sec:heterogeneous}.

A hashing accelerator can be used to provide accelerated hashing capabilities
to a processing element in such an architecture, or it can be implemented as
a processing element by itself.

Other parts of the processor can then offload hashing operations to the
accelerator tiles or the accelerated processor elements and continue
working on other tasks or simply shut down, saving energy.

Using a DMA module can allow further
power savings, as the processor does not have to do data transfer to
and from the hashing accelerator by itself and can spend its time
in a low-power sleep state instead.

\section{Outline of the report}
\todo{Kristian: Som jeg har nevnt tidligere, outlining av rapporten er noe jeg har sett gått igjen i andre artikler i TDT1. Du kan enden se over og godkjenne (evt forbedre?), eller hviske ut. -Torbjørn}
Chapter \ref{cha:background} introduces the theory, ideas and issues related to bitcoin mining, SHA-256 hashing and DMA, and how they should fit into SHMAC.
Chapter \ref{cha:related-work} covers related work, which have given valuble insight for this work.
Chapter \ref{cha:architecture} presents the design and architecture for the hashing module and the DMA module.
Chapter \ref{cha:evaluation} presents how the hashing module and the DMA were tested, and the results.
Chapter \ref{cha:conclusion} concludes the project and lists up future work.

In addition, \ref{app:DMA-arch} goes into the details behind the DMA architecture at a greater length.