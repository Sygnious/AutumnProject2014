\chapter{Introduction}

This report details an exploration of the possible benefits of accelerating
SHA-256 hashing using an accelerator module and a DMA, for possible future
inclusion in the SHMAC\footnote{\textbf{S}ingle-ISA \textbf{H}eterogenous
\textbf{Ma}ny-core \textbf{C}omputer} research computer.

\section{Assignment Text}

This project aims to develop an accelerator for a multi-core heterogeneous
computing platform instantiating a hardware-based Bitcoin miner. Bitcoin mining
is, at its core, a SHA-256 hashing problem, so part of the assignment will be to
make the interface generic enough such that other cryptographic algorithms can
be readily developed. Furthermore, various constraints on the heterogeneous
architectures may necessitate the implementation of new features, for example,
direct memory access (DMA). If time permits the student is to implement other
necessary architectural improvements or dependencies.

Specifically, the following points are envisioned:

\begin{enumerate}
\item A literature review discussing possible benefits of a bitcoin miner in the
      context of a multi-core heterogeneous computing platform.
\item Implement a SHA-256 accelerator in hardware/on an FPGA (an other hashing algorithms if time permits).
\item Wrap the accelerator into a package (for example, a SHMAC tile if time permits and current ISA-issues permit).
\item Evaluation of the energy efficiency of the proposed tile compared to a general-purpose CPU.
\item Implementing a generic DMA interface for SHMAC and evaluation of the impact on the accelerator.
\item Investigation of heterogeneous mapping possibilities (for example, getting the CPU and accelerator to cooperate on different parts).
\end{enumerate}

\section{Interpretation}

\section{Bitcoin Mining as a Model Problem}

Bitcoin is a virtual currency that is produced through an algorithm that
requires repeatedly hashing data structures with the SHA-256 algorithm
in order to produce a target hash value. There are more details about
bitcoins and how they are created in section \ref{sec:bitcoins}.

Bitcoin mining is a problem that there exists many different accelerators
for, implemented both in software and hardware, in FPGAs and ASICs\cite{bespoke-silicon}.
This gives a good opportunity for comparing the performance of our
accelerator with other implementations.

There is also an established performance metric for bitcoin mining hardware:
hashes per second, often abbreviated H/s. This metric makes it easy
to measure and compare the performance of our accelerator.

\section{Accelerators and Heterogeneous Systems}

A heterogeneous processor architecture is an architecture that contains multiple
cores of different sizes and capabilities. For an overview of why heterogeneous
architectures may become increasingly important in the future, see section
\ref{sec:heterogeneous}.

A hashing accelerator can be used to provide accelerated hashing capabilities
to a processing element in such an architecture, or it can be implemented as
a processing element by itself.

Other parts of the processor can then offload hashing operations to the
accelerator tiles or the accelerated processor elements and continue
working on other tasks or simply shut down, saving energy.

